\chapter{Conclusions}

The goal of detecting and locating small differences between images
and volumes presents a very interesting problem, not only limited to
the medical field. It is still a big challenge today for which a
complete solution is required.\\

Even though in our specific case we deal with patients that have
received mild trauma, it can still mean a lot for the quality of life
of a person if he or she receives help when needed, and if the
physician is able to locate the exact area affected by the injury.\\

The applications presented in this project were created as modules of
a bigger and already existent application called \textit{3D Slicer},
which contains many other functionalities for medical image analysis
and manipulation.

In this way, the applications are not only given an interface that is
easy to use, but also allow the user to utilize any of the
other modules already present within \textit{3D Slicer}.\\

The applications attempt to solve the problem for MRI volumes with two
different techniques, both obtained from analysis of previous research
done in the area.

The results obtained with the voxel-based method are quite good, even
when the artificial differences added where too small to be found with
the naked eye. During this research, this method was successfully used
in a medical study with real patients performed by the Uppsala
University Hospital.

The tensor-based method also produced useful results for differences
of large and medium size. Even though the results were not as expected
for smaller differences, the method could be the start for a very
interesting future development possibly using the \textit{Jacobian
  matrix}.\\

Finally, it is important to highlight that the results of both methods
as they are now, are just a suggestion of the size and location of the
differences between the volumes; as a consequence, the procedure can
not be completely automatic, since the applications still need the
help of a medical practitioner in order to define the differences more
exactly and possibly rule out errors.


\section{Future works}

The tensor-based method needs to be improved in order to obtain more
exact results, and for it to be able to detect smaller differences.

The first thing that should be done is either find a better way of
figuring out which range of values of the \textit{Jacobian
  determinant} are actually useful to detect differences, or directly
using the \textit{Jacobian matrix} to calculate the changes in volume
between the MRIs.

Manipulating the \textit{Jacobian matrix} is harder than using the
\textit{Jacobian determinant} since, according to my research, there
is no \textit{ITK} function or library already implemented to use
it. Some theoretical pointers on how to use the matrix can be found in
\cite{ashburner}.\\

The voxel-based method seems to work quite well in our experiments
with both real and artificial differences between volumes. However, it
would be very interesting to be able to compare its results against
another tool with the same goals.\\

We would like the modules to be available for all the users of
\textit{3D Slicer} and for the public in general. To achieve this
goal, a few specific conditions must be fulfilled in order for the
module to be accepted as part of the \textit{3D Slicer} code.

This would probably produce some criticism from the \textit{3D Slicer}
developer community, which could contain useful comments on the
current implementation of the modules and how to improve it.\\


