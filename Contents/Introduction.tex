\chapter{Introduction}

\section{Description of the problem}
When a person suffers an accident that produces trauma to the head, one of the first things that should be performed is an MRI of the brain region in order to spot possible brain injuries. If the damage is large, a medical specialist can usually recognize which parts of the brain have been affected without automated help. However, if the damage is minor, it is generally much harder for the doctor to figure out what could be the consequences of the injury. \\

For this particular project we have the following conditions: A patient's brain is scanned, producing a first MRI just after receiving trauma to the head. The patient has not received extensive damage, and so the MRI comes out as expected from a medical specialist. After a period of three to twelve months later, a new MRI of the same patient is taken and is compared with the initial MRI.

It is assumed that if there are symptoms produced as a consequence of the accident, then there must be differences in the brain of the patient, even when the differences might be too small to see with the naked eye.\\


The goal of this project is to analyse and test different registration and morphometry methods, in order to compare this type of brain MRIs and produce information on the differences between them. 

The final result of the project should be a tool that allows the medical specialist to do the comparison on his or her own.


\section{Importance}

Manual examination of MRI studies suffers from many problems. The study can be affected specially by acquisition related factors such as:
\begin{itemize}
\item One cannot assume a one-to-one correspondence between slices from one acquisition to the next in order to make side-by-side comparisons.
\item A different scanner may be used in every examination, producing a scan which will have different signal characteristics.
\item Often, scanning parameters are not the same form one acquisition to the next.
\item Change can present itself in many ways. The radiologist is required to assimilate all this data before making a decision, which often is quite difficult.
\end{itemize}

A very important part of change detention is therefore not simply the detection of change but the separation of acquisition-related change from disease-related change. Also, methods that produce objective, reproducible and accurate metrics of disease course are of great interest since the change in appearance over time is essential to understanding disease course \cite{review1}.

Also, it is not enough to be able to determine if there are differences, since from a clinical standpoint, knowing where and how changes have occurred is as important as knowing that they have occurred \cite{review1}.\\


From a practical point of view, it is also very important for the patients to be able to obtain a correct diagnosis, specially if the symptoms they are suffering are hard to explain or subjective, as it is common with non-extensive brain injuries.

\section{Related Works}
Many related works focus on inter-subject studies, which means that they compare images acquired from different individuals in order to diagnose diseases or identify abnormalities. This generally implies that the first step in the analysis process is one of \textit{spatial normalization} or \textit{inter-subject registration}, in which the aim is to reduce the anatomical variability in the volumetric brain scans.

The works of \cite{zeffiro,svensen} use this technique combined with voxelwise group analysis of functional magnetic resonance imaging (fMRI), and the works of \cite{ardekani1,jones} also use spatial normalization combined with diffusion tensor imaging in order to study brain white matter.\\

More similar works, related to the quantification of small changes in volume observed over time can be observed in \cite{holden,rey}. The results of these works are used to diagnose and evaluate disease progression and treatment.


Some other works that also use the specific techniques described later in this project are:
\begin{itemize}
\item Hajnal et al. in \cite{hajnal} and Lemieux et al. in \cite{lemieux} utilize subtraction after applying rigid registration on the volumes.
\item Rey et al. in \cite{rey} introduce the Jacobian operator of the deformation field resulting from the registration as a measure of local volume variation.
\item Pohl et al. in \cite{pohlCT} describe a semiautomatic procedure targeted toward identifying difficult-to-detect changes in brain tumor imaging. The result of this study is also a module for \textit{3D Slicer}, and was very useful for this project since its source code is open for other researchers to view. 

More information about the module can be found on its webpage: 

\url{http://www.slicer.org/slicerWiki/index.php/Documentation/4.1/Modules/ChangeTracker}
\end{itemize}


