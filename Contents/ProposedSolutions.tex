\chapter{Proposed Solutions}

\section{Tools used}
\subsection{3D Slicer}
\textbf{Slicer}, or \textbf{3D Slicer}, is a free, open source software package for visualization and image analysis. It is natively designed to be available on multiple platforms, including Windows, Linux and Mac Os X.

3D Slicer provides image registration, processing of DTI (diffusion tractography), an interface to external devices for image guidance support, and GPU-enabled volume rendering, among other capabilities. 3D Slicer has a modular organization that allows the easy addition of new functionality and provides a number of generic features not available in competing tools.

3D Slicer is built on VTK, a pipeline-based graphical library that is widely used in scientific visualization. In version 4, the core application is implemented in C++, and the API is available through a Python wrapper to facilitate rapid, iterative development and visualization in the included Python console. The user interface is implemented in Qt, and may be extended using either C++ or Python.

Slicer supports several types of modular development. Fully interactive, custom interfaces may be written in C++ or Python. Command-line programs in any language may be wrapped using a light-weight XML specification, from which a graphical interface is automatically generated \cite{slicer}.

For more information on this tool please refer to its official webpage: 

\url{http://www.slicer.org/}

\subsection{ITK}
ITK stands for \textbf{Insight Segmentation and Registration Toolkit}, it's a cross-platform, open-source application development framework widely used for the development of image segmentation and image registration programs.

ITK  is implemented in C++ and it is wrapped for Tcl, Python and Java. This enables developers to create software using a variety of programming languages.

ITK's code is highly efficient, which means that many software problems are discovered at compile-time, rather than at run-time during program execution. It also enables ITK to work on two, three, four or more dimensions.

For more information on this tool please refer to its official webpage: 

\url{http://www.itk.org/}

\subsection{Other Tools}
\subsubsection{Programming Languages}
The programming languages chosen during this project are \textbf{C++} and \textbf{Python}, mainly because they are main languages in which 3D Slicer is written, which means that it was easier to communicate with 3D Slicer by using them.

\subsubsection{MATLAB}
MATLAB is a numerical computing environment and programming language. It allows matrix manipulations, plotting of functions and data, implementation of algorithms, creation of user interfaces, and interfacing with programs written in other languages, including C, C++, Java, and Fortran \cite{matlab}.

MATLAB was used during this project specifically to create and quickly manipulate MRI volume files.

Official website for this tool: \url{http://www.mathworks.com/}

\subsubsection{ParaView}
ParaView is an open-source, multi-platform data analysis and visualization application. ParaView users can quickly build visualizations to analyze their data using qualitative and quantitative techniques.

ParaView was used during this project to visualize the deformation fields produced after the registration of two volumes.

Official website for this tool: \url{http://www.paraview.org/}

\section{Voxel-based method}

\section{Tensor-based method}