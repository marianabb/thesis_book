\chapter{Applications}

The original idea for the project came out of the necessity for a
program that could be used to find small differences between MRIs of
trauma patients at the Uppsala University Hospital.

The main doctor interested in the project is PhD. Raili Raininko, from
the Department of Radiology at the hospital, who collaborated with her
experience and comments during the entire course of this project.\\

So far, the application has been used in a study made with the
collaboration of the Department of Neuroscience of Uppsala University
and the Department of Radiology of the Uppsala University Hospital.

During the mentioned study, the MRIs of nineteen patients who
presented mild traumatic brain injuries where compared; with the first
MRI taken 2 or 3 days after the injury, and the second 3 to 7 months
after.

The application created during this project was used to corroborate
the results obtained after visual analysis of the MRIs by an expert
physician. 

The study concludes that loss of brain volume may be a
feasible marker of brain pathology after mild traumatic brain
injuries.\\

Volume comparison, and specifically medical image comparison,
continues to be a very challenging problem. The applications developed
during this project should be well received by the medical audience,
even if they do not guarantee a complete automatization of the
comparison process.

As described later on the \textit{Future Works} section, it would be
very useful to make the modules implemented available to the public by
making them a part of a new release of \textit{3D Slicer}.

In this way, the modules could be used, commented on, and even
possibly improved, by anyone who would be willing to use them or by a
member of the \textit{3D Slicer} developer community.


% TODO Joel project??