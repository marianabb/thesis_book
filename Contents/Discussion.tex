\chapter{Discussion}
This chapter contains an a analysis of the strenghts and weaknesses
of each of the proposed methods based on the results obtained after
several experiments with patient containing real and artificially
added differences.

\section{Voxel-based method}
This method performs especially well in cases of volume loss, since
this condition implies larger differences in intensity between both
volumes. According to the experiments performed, the size of this
differences can be really small and the method may still produce
useful results.\\

The method depends a lot on the registration results obtained in the
second step. If the registration result is poor, the program will
produce ``ghosts'' or false differences that may confuse the user.

Note that a poor registration result might not necessarily be a direct
consequence of the registration method chosen, it may also be due to
problems with the volumes; for example, if the patient's position
changes a lot from one volume to the next, or if the MRI machine has
very distinct settings in each examination.\\

Since in our specific case we deal with differences that are quite
small, sometimes it is still hard to see them in the results;
specially if we are looking at a screenshot of the application, rather
than the actual application where we can see all of the frames in the
volume.

\section{Tensor-based method}
This method has a lot of potential; since in theory, by using the
deformation field resulting from the registration, we have all the
information to be able to identify all the differences between the
volumes.\\

In spite of its potential, further analysis of the method results in
the following challenges:
\begin{enumerate}
\item We need to generate a ``useful'' deformation field; which in
  this case means a deformation field that is not too noisy, and from
  which we can obtain real differences.
\item We also need to figure out ``good values'' for the percentages
  of growth and shrinkage that we are going to use for the final
  result. This point is particularly complicated, since the
  \textit{Jacobian determinant} values are usually not distributed in
  the same way for different volumes.\\
\end{enumerate}

The module attempts to help the user find the right values by making
the most relevant numbers interactive in the user interface. However,
it is sometimes very hard to distinguish between actual differences
and noise caused by a non smooth deformation field.\\

Also, the \textit{Jacobian determinant} values are not regularly
distributed around zero; this means that in most cases, the range of
the values representing growth (located above zero), is much larger
that the range of values that represent shrinkage (values below
zero). As a consequence, the percentage of shrinkage is much more
sensitive to small changes than the percentage of growth.
